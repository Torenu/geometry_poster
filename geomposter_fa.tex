\documentclass[17pt, a0paper, landscape]{tikzposter} 
\tikzposterlatexaffectionproofoff % حذف واترمارک

\usepackage{amsmath,amssymb}
\usepackage[shortlabels]{enumitem}
\usepackage{graphicx}
\usepackage{parskip}
\usepackage[unicode]{hyperref}
\usepackage{pgfplots}
\pgfplotsset{compat=1.15}
\usepackage{mathrsfs}
\usetikzlibrary{arrows}

% تنظیم رنگ‌ها
\usepackage{xcolor}
\usepackage[normalem]{ulem}
\useunder{\uline}{\ulined}{}%
\DeclareUrlCommand{\bulurl}{\def\UrlFont{\ttfamily\color{blue}\ulined}}

% --- تنظیمات فارسی و راست‌چین ---
% بسته زی‌پرشین باید آخرین بسته باشد
\usepackage{xepersian}

% تنظیم فونت‌ها (نام فونت را بر اساس فونت‌های نصب شده روی سیستم خود تغییر دهید)
% اگر فونت Yas ندارید، از Tahoma یا B Nazanin استفاده کنید
\settextfont{XB Niloofar} 
\setlatintextfont{Liberation Serif}
\setdigitfont{Yas}

% تعریف رنگ‌های پوستر
\definecolorpalette{PurpleGrayBlue}{
    \definecolor{colorOne}{HTML}{D40279}
    \definecolor{colorTwo}{HTML}{7F8897}
    \definecolor{colorThree}{HTML}{006C9E}
}

% تنظیمات عنوان
\settitle{\centering \@title}
\title{\textcolor[HTML]{D40279}{\textbf{\Huge{آمادگی برای آزمون ریاضی: هندسه مسطحه}}}}

\makeatletter
\newcommand\insertlogoi[2][]{\def\@insertlogoi{\includegraphics[#1]{#2}}}
\newcommand\insertlogoii[2][]{\def\@insertlogoii{\includegraphics[#1]{#2}}}
\newlength\LogoSep
\setlength\LogoSep{0pt}

% تنظیم تصاویر لوگو (مسیر فایل‌ها را بررسی کنید)
\insertlogoi[width=10cm]{figures/logo.pdf} 
\insertlogoii[width=5cm]{example-image-b}

\renewcommand\maketitle[1][]{  
    \normalsize
    \setkeys{title}{#1}
    \node[transparent,inner sep=\TP@titleinnersep, line width=\TP@titlelinewidth, anchor=north, minimum width=\TP@visibletextwidth-2\TP@titleinnersep]
        (TP@title) at ($(0, 0.5\textheight-\TP@titletotopverticalspace)$) {\parbox{\TP@titlewidth-2\TP@titleinnersep}{\TP@maketitle}};
    \draw let \p1 = ($(TP@title.north)-(TP@title.south)$) in node {
        \setlength{\TP@titleheight}{\y1}
        \setlength{\titleheight}{\y1}
        \global\TP@titleheight=\TP@titleheight
        \global\titleheight=\titleheight
    };

    \setlength{\titleposleft}{-0.5\titlewidth}
    \setlength{\titleposright}{\titleposleft+\titlewidth}
    \setlength{\titlepostop}{0.5\textheight-\TP@titletotopverticalspace}
    \setlength{\titleposbottom}{\titlepostop-\titleheight}

    \TP@titlestyle

    \node[inner sep=\TP@titleinnersep, line width=\TP@titlelinewidth, anchor=north, minimum width=\TP@visibletextwidth-2\TP@titleinnersep]
        at (0,0.5\textheight-\TP@titletotopverticalspace)
        (title)
        {\parbox{\TP@titlewidth-2\TP@titleinnersep}{\begin{center}\TP@maketitle\end{center}}}; % Center title content

    \node[inner sep=400pt,anchor=west] 
      at ([xshift=-\LogoSep]title.west)
      {\@insertlogoi};

    \node[inner sep=400pt,anchor=east] 
      at ([xshift=\LogoSep]title.east)
      {\Huge\bulurl{https://plucik.ru/}};

    \normalsize
    \setlength{\TP@blocktop}{\titleposbottom-\TP@titletoblockverticalspace}
}
\makeatother

\usetheme{Simple}
\colorlet{titlebgcolor}{white}
\usetitlestyle{Default}
\useblockstyle{TornOut}

% علامت توازی
\newcommand{\parallelsum}{\mathbin{\|}}

\begin{document}

% اجبار به راست‌چین بودن کل سند
\begin{RTL}

% تعریف رنگ‌ها برای تصاویر احتمالی
\definecolor{atfczz}{rgb}{0,0.6,0}
\definecolor{fruycc}{rgb}{0.8313725490196079,0.00784313725490196,0.4745098039215686}
\definecolor{qqwuqq}{rgb}{0,0.6,0}
\definecolor{ududff}{rgb}{0.8313725490196079,0.00784313725490196,0.4745098039215686}
\definecolor{qqqqff}{rgb}{0.8313725490196079,0.00784313725490196,0.4745098039215686}
\definecolor{atfcqq}{rgb}{0,0.6,0}
\definecolor{xdxdff}{rgb}{0.8313725490196079,0.00784313725490196,0.4745098039215686}
\definecolor{duqsxz}{rgb}{0.8313725490196079,0.00784313725490196,0.4745098039215686}

\maketitle

\begin{columns}
% ستون اول (راست)
\column{0.166}
\block{قضیه تالس}
{
$$l_1 \ \parallelsum \ l_2 \ \parallelsum \ l_3 \iff \frac {a} {a'} = \frac {b} {b'} = \frac {c} {c'}$$
\begin{center}
\input{figures/thales_plain.tex}
\end{center}
}
\block{نقطه برخورد میانه‌ها}
{
$$\frac {CO} {OM} = \frac {AO} {OD} = \frac {BO} {OE} = \frac {2} {1}$$
\begin{center}
\input{figures/median_plain.tex}
\end{center}
}
\block{ویژگی خاص ذوزنقه}
{
\begin{center}
$F, M$ وسطِ قاعده‌های ذوزنقه $ABCD$\\
$\Rightarrow$ نقاط $E, F, M$ هم‌خط هستند \\
\input{figures/trapezoid_plain.tex}
\end{center}
}
\block{قضیه فیثاغورس}
{
$$\Delta ABC \text{ قائم‌الزاویه } \iff a^2 + b^2 = c^2$$
\begin{center}
\input{figures/pythagoras_plain.tex}
\end{center}
}
\block{ارتفاع در مثلث قائم‌الزاویه}
{
$$h^2 = x \cdot y$$
\begin{center}
\input{figures/h_right_plain.tex}
\end{center}
}

% ستون دوم
\column{0.166}
\block{قضیه سینوس‌ها}
{
$$\frac {a} {\sin{\alpha}} = \frac {b} {\sin{\beta}} = \frac {c} {\sin{\gamma}} = 2R$$
\begin{center}
\input{figures/sin_plain.tex}
\end{center}
}
\block{قضیه کسینوس‌ها}
{
$$c^2 = a^2 + b^2 - 2ab\cdot\cos{\gamma}$$
\begin{center}
\input{figures/cos_plain.tex}
\end{center}
}
\block{قضیه منلائوس}
{
$$\frac {AN} {NB} \cdot \frac {BM} {MC} \cdot \frac {CK} {KA} = 1$$
\begin{center}
\input{figures/menelai_plain.tex}
\end{center}
}
\block{قضیه سوا (Ceva)}
{
$$\frac {AM} {MB} \cdot \frac {BN} {NC} \cdot \frac {CK} {KA} = 1$$
\begin{center}
\input{figures/cheva_plain.tex}
\end{center}
}
\block{قضیه وان اوبل}
{
$$\frac {BO} {OK} = \frac {BN} {NC} + \frac {BM} {MA}$$
\begin{center}
\input{figures/van-obel_plain.tex}
\end{center}
}

% ستون سوم
\column{0.166}
\block{قضیه نیمساز}
{
$$\frac {a} {b} = \frac {x} {y}$$
\begin{center}
\input{figures/bisector_plain.tex}
\end{center}
}
\block{مساحت مثلث}
{
$$S_\Delta = \frac 1 2 \cdot ah = \frac 1 2 \cdot ab \cdot \sin{\alpha}$$
\begin{center}
\input{figures/triangle_area_plain.tex}
\end{center}
}
\block{فرمول هرون}
{
$$p = \frac {a + b + c} {2},$$
$$S_\Delta = \sqrt{p(p-a)(p-b)(p-c)}$$
\vspace{0.5cm}
\begin{center}
\input{figures/geron_plain.tex}
\end{center}
}
\block{مساحت ذوزنقه}
{
$$S_{ABCD} = \frac {AD + BC} {2}\cdot h$$
\begin{center}
\input{figures/trapezoid_area_plain.tex}
\end{center}
}
\block{مساحت چهارضلعی}
{
$$S_{ABCD} = \frac 1 2 \cdot AC \cdot BD \cdot \sin{\alpha}$$
\begin{center}
\input{figures/4p_area_plain.tex}
\end{center}
}

% ستون چهارم
\column{0.166}
\block{نسبت مساحت‌ها}
{
(با زاویه مشترک)
$$\frac {S_{\Delta ABC}} {S_{\Delta ADE}} = \frac {AB} {AD} \cdot \frac {AC} {AE}$$
\begin{center}
\input{figures/area_frac_plain.tex}
\end{center}
}
\block{زوایای محاطی}
{
$$\angle{AEB} = \angle{ADB} = \frac 1 2 \smallsmile{AB}$$
\begin{center}
\input{figures/duga_plain.tex}
\end{center}
}
\block{زاویه مماس و وتر}
{
$$\angle{ABC} = \angle{ADB} = \frac 1 2 \smallsmile{AB}$$
\begin{center}
\input{figures/kas_angle_plain.tex}
\end{center}
}
\block{زاویه بین دو وتر}
{
$$\angle{AKB} = \angle{EKD} = \frac {\smallsmile{AB} \ + \smallsmile{ED}} {2}$$
\begin{center}
\input{figures/chord_angle_plain.tex}
\end{center}
}
\block{زاویه بین دو قاطع}
{
$$\angle{BAC} = \frac {\smallsmile {DE} \ - \smallsmile{BC}} {2}$$
\begin{center}
\input{figures/sec_angle_plain.tex}
\end{center}
}

% ستون پنجم
\column{0.166}
\block{زوایا در خطوط موازی}
{
$$k \ \parallelsum \ l \iff \alpha = \beta = \gamma$$
\begin{center}
\input{figures/parallel_plain.tex}
\end{center}
}
\block{دو مماس از یک نقطه}
{
$$AB = AC$$
\begin{center}
\input{figures/2kas_plain.tex}
\end{center}
}
\block{رابطه مماس و قاطع}
{
$$\Delta ABC \sim \Delta ADB, \quad AB^2 = AC \cdot AD$$
\begin{center}
\input{figures/kas_plain.tex}
\end{center}
}
\block{وترهای متقاطع}
{
$$\Delta AKB \sim \Delta EKD, \quad AK \cdot KD = BK \cdot KE$$
\vspace{0.5cm}
\begin{center}
\input{figures/chord_plain.tex}
\end{center}
}
\block{رابطه دو قاطع}
{
$$\Delta ABC \sim \Delta ADE, \quad AB \cdot AD = AC \cdot AE$$
\begin{center}
\input{figures/sec_plain.tex}
\end{center}
}

% ستون ششم
\column{0.166}
\block{دایره محاطی مثلث}
{
\begin{center}
مرکز: محل برخورد نیمسازهای داخلی \\
\vspace{0.5cm}
\input{figures/cvt_plain.tex}
\end{center}
}
\block{دایره محیطی مثلث}
{
\begin{center}
مرکز: محل برخورد عمودمنصف‌ها \\
\vspace{0.5cm}
\input{figures/cot_plain.tex}
\end{center}
}
\block{محیطی (مثلث قائم‌الزاویه)}
{
\begin{center}
$\Delta ABC$ قائم‌الزاویه $\iff AC$ قطر است،\\
AO = OC = OB = R\\
\vspace{0.5cm}
\input{figures/ort_plain.tex}
\end{center}
}
\block{چهارضلعی محاطی}
{
\begin{center}
$ABCD$ محاطی $\iff \alpha + \beta = 180^\circ$\\
\vspace{0.5cm}
\input{figures/v4_plain.tex}
\end{center}
}
\block{چهارضلعی محیطی}
{
\begin{center}
$ABCD$ محیطی $\iff AB + CD = BC + AD$\\
\vspace{0.5cm}
\input{figures/o4_plain.tex}
\end{center}
}

\end{columns}
\end{RTL}
\end{document}