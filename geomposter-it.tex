\documentclass[17pt, a0paper, landscape]{tikzposter} % See Section 3
\tikzposterlatexaffectionproofoff % disable watermark

%\usepackage[T2A]{fontenc} %used for cyrillic font
\usepackage[T1]{fontenc} % used for western fonts
\usepackage[utf8]{inputenc}
\usepackage{polyglossia}
\setmainlanguage{russian}
%\setotherlanguage{english}
\setotherlanguage{italian} % added italian language
%\setkeys{russian}{babelshorthands=true}

\usepackage{amsmath,amssymb}
\usepackage[shortlabels]{enumitem}
\usepackage{graphicx}
\usepackage{comment}
\usepackage{parskip}
\usepackage[unicode]{hyperref}
\AddEnumerateCounter{\asbuk}{\russian@asbuk@alph}{}
\usepackage{pgfplots}
\pgfplotsset{compat=1.15}
\usepackage{mathrsfs}
\usetikzlibrary{arrows}

% \bulurl command to make blue hyperlink
\usepackage{xcolor}
\usepackage[normalem]{ulem}
\useunder{\uline}{\ulined}{}%
\DeclareUrlCommand{\bulurl}{\def\UrlFont{\ttfamily\color{blue}\ulined}}

% changing font
\usepackage{PTSans}

\setmainfont{PT Sans}
\setromanfont{PT Sans} 
\setsansfont{PT Sans} 
\setmonofont{Consolas} 

\newfontfamily{\cyrillicfont}{PT Sans} 
\newfontfamily{\cyrillicfontrm}{PT Sans}
\newfontfamily{\cyrillicfonttt}{PT Sans}
\newfontfamily{\cyrillicfontsf}{PT Sans}

\usepackage{unicode-math}
\setmathfont{Fira Math}

% define tikzposter colors (mainly 1st one)
\definecolorpalette{PurpleGrayBlue}{
    \definecolor{colorOne}{HTML}{D40279}
    \definecolor{colorTwo}{HTML}{7F8897}
    \definecolor{colorThree}{HTML}{006C9E}
}

% custom title configuration
\settitle{\centering \@title}
%\title{\textcolor[HTML]{D40279}{\textbf{\Huge{Preparation for the State Exam in Mathematics: Plane Geometry}}}}
%\title{\textcolor[HTML]{D40279}{\textbf{\Huge{Preparazione per  l'esame di stato in matematica: Geometria Piana}}}} - literal translation
\title{\textcolor[HTML]{D40279}{\textbf{\Huge{Preparazione per la verifica di matematica: Geometria Piana}}}} % terms most common used in italy


% the following bit of code adds logo blocks on the left and right to the title
\makeatletter
\newcommand\insertlogoi[2][]{\def\@insertlogoi{\includegraphics[#1]{#2}}}
\newcommand\insertlogoii[2][]{\def\@insertlogoii{\includegraphics[#1]{#2}}}
\newlength\LogoSep
\setlength\LogoSep{0pt}

\insertlogoi[width=10cm]{figures/logo.pdf}
\insertlogoii[width=5cm]{example-image-b} % I inserted my hyperlink instead of the logo on the right so this line is not used

\renewcommand\maketitle[1][]{  % #1 keys
    \normalsize
    \setkeys{title}{#1}
    % Title dummy to get title height
    \node[transparent,inner sep=\TP@titleinnersep, line width=\TP@titlelinewidth, anchor=north, minimum width=\TP@visibletextwidth-2\TP@titleinnersep]
        (TP@title) at ($(0, 0.5\textheight-\TP@titletotopverticalspace)$) {\parbox{\TP@titlewidth-2\TP@titleinnersep}{\TP@maketitle}};
    \draw let \p1 = ($(TP@title.north)-(TP@title.south)$) in node {
        \setlength{\TP@titleheight}{\y1}
        \setlength{\titleheight}{\y1}
        \global\TP@titleheight=\TP@titleheight
        \global\titleheight=\titleheight
    };

    % Compute title position
    \setlength{\titleposleft}{-0.5\titlewidth}
    \setlength{\titleposright}{\titleposleft+\titlewidth}
    \setlength{\titlepostop}{0.5\textheight-\TP@titletotopverticalspace}
    \setlength{\titleposbottom}{\titlepostop-\titleheight}

    % Title style (background)
    \TP@titlestyle

    % Title node
    \node[inner sep=\TP@titleinnersep, line width=\TP@titlelinewidth, anchor=north, minimum width=\TP@visibletextwidth-2\TP@titleinnersep]
        at (0,0.5\textheight-\TP@titletotopverticalspace)
        (title)
        {\parbox{\TP@titlewidth-2\TP@titleinnersep}{\TP@maketitle}};

    \node[inner sep=400pt,anchor=west] 
      at ([xshift=-\LogoSep]title.west)
      {\@insertlogoi};

    \node[inner sep=400pt,anchor=east] 
      at ([xshift=\LogoSep]title.east)
      {\Huge\bulurl{https://plucik.ru/}}; %\@insertlogoii if you want to have 2nd logo on the right

    % Settings for blocks
    \normalsize
    \setlength{\TP@blocktop}{\titleposbottom-\TP@titletoblockverticalspace}
}
\makeatother
\usetheme{Simple} % See Section 5
\colorlet{titlebgcolor}{white}
\usetitlestyle{Default}
\useblockstyle{TornOut}

% parallel sign
\newcommand{\parallelsum}{\mathbin{\|}}

\begin{document}

% define colors for geogebra imported tikz images
\definecolor{atfczz}{rgb}{0,0.6,0}
\definecolor{fruycc}{rgb}{0.8313725490196079,0.00784313725490196,0.4745098039215686}
\definecolor{qqwuqq}{rgb}{0,0.6,0}
\definecolor{ududff}{rgb}{0.8313725490196079,0.00784313725490196,0.4745098039215686}
\definecolor{qqqqff}{rgb}{0.8313725490196079,0.00784313725490196,0.4745098039215686}
\definecolor{atfcqq}{rgb}{0,0.6,0}
\definecolor{xdxdff}{rgb}{0.8313725490196079,0.00784313725490196,0.4745098039215686}
\definecolor{duqsxz}{rgb}{0.8313725490196079,0.00784313725490196,0.4745098039215686}

\maketitle

% main body of the document, pretty straightforward
\begin{columns}
	\column{0.166}
	%\block{Thales' Theorem}
	\block{Teorema di Talete}
	{
		$$l_1 \ \parallelsum \ l_2 \ \parallelsum \ l_3 \iff \frac {a} {a'} = \frac {b} {b'} = \frac {c} {c'}$$
		\begin{center}
			\input{figures/thales_plain.tex}
		\end{center}
	}
	%\block{Centroid of a Triangle}
	\block{Baricentro di un Triangolo}
	{
		$$\frac {CO} {OM} = \frac {AO} {OD} = \frac {BO} {OE} = \frac {2} {1}$$
		\begin{center}
			\input{figures/median_plain.tex}
		\end{center}
	}
	%\block{Remarkable Property of a Trapezoid}
	\block{Proprietà Notevoli di un Trapezio}
	{
		\begin{center}
			$F, M$ - punti medi delle basi del Trapezio $ABCD$\\
			$\Rightarrow$ i punti $E, F, M$ giacciono sulla stessa linea \\
			\input{figures/trapezoid_plain.tex}
		\end{center}
	}
	%\block{Pythagorean Theorem}
	\block{Teorema di Pitagora}
	{
		$$\Delta ABC \text{ - Angolo Retto } \iff a^2 + b^2 = c^2$$
		\begin{center}
			\input{figures/pythagoras_plain.tex}
		\end{center}
	}
	%\block{Altitude in a Right Triangle}
	\block{Altezza di un Triangolo Rettangolo}
	{
		$$h^2 = x \cdot y$$
		\begin{center}
			\input{figures/h_right_plain.tex}
		\end{center}
	}
	
	\column{0.166}
	%\block{Law of Sines}
	\block{Teorema dei Seni}
	{
		$$\frac {a} {\sin{\alpha}} = \frac {b} {\sin{\beta}} = \frac {c} {\sin{\gamma}} = 2R$$
		\begin{center}
			\input{figures/sin_plain.tex}
		\end{center}
	}
	%\block{Law of Cosines}
	\block{Teorema dei Coseni}
	{
		$$c^2 = a^2 + b^2 - 2ab\cdot\cos{\gamma}$$
		\begin{center}
			\input{figures/cos_plain.tex}
		\end{center}
	}
	%\block{Menelaus' Theorem}
	\block{Teorema di Menelao}
	{
		$$\frac {AN} {NB} \cdot \frac {BM} {MC} \cdot \frac {CK} {KA} = 1$$
		\begin{center}
			\input{figures/menelai_plain.tex}
		\end{center}
	}
	%\block{Ceva's Theorem}
	\block{Terorema di Ceva}
	{
		$$\frac {AM} {MB} \cdot \frac {BN} {NC} \cdot \frac {CK} {KA} = 1$$
		\begin{center}
			\input{figures/cheva_plain.tex}
		\end{center}
	}
	%\block{Van Aubel's Theorem}
	\block{Terorema di Van Aubel}
	{
		$$\frac {BO} {OK} = \frac {BN} {NC} + \frac {BM} {MA}$$
		\begin{center}
			\input{figures/van-obel_plain.tex}
		\end{center}
	}
	
	\column{0.166}
	%\block{Angle Bisector Theorem}
	\block{Teorema della Bisettrice}
	{
		$$\frac {a} {b} = \frac {x} {y}$$
		\begin{center}
			\input{figures/bisector_plain.tex}
		\end{center}
	}
	%\block{Area of a Triangle}
	\block{Area di un Triangolo}
	{
		$$S_\Delta = \frac 1 2 \cdot ah = \frac 1 2 \cdot ab \cdot \sin{\alpha}$$
		\begin{center}
			\input{figures/triangle_area_plain.tex}
		\end{center}
	}
	%\block{Heron's Formula}
	\block{Formula di Erone}
	{
		$$p = \frac {a + b + c} {2},$$
		$$S_\Delta = \sqrt{p(p-a)(p-b)(p-c)}$$
		\vspace{0.5cm}
		\begin{center}
			\input{figures/geron_plain.tex}
		\end{center}
	}
	%\block{Area of a Trapezoid}
	\block{Area di un Trapezio}
	{
		$$S_{ABCD} = \frac {AD + BC} {2}\cdot h$$
		\begin{center}
			\input{figures/trapezoid_area_plain.tex}
		\end{center}
	}
	%\block{Area of a Quadrilateral}
	\block{Area di un Quadrilatero}
	{
		$$S_{ABCD} = \frac 1 2 \cdot AC \cdot BD \cdot \sin{\alpha}$$
		\begin{center}
			\input{figures/4p_area_plain.tex}
		\end{center}
	}
	
	\column{0.166}
	%\block{Ratio of the Areas of Two Triangles with a Common Angle}
	\block{Teorema sul rapporto delle aree di due triangoli aventi gli stessi angoli (triangoli simili)}
	{
		$$\frac {S_{\Delta ABC}} {S_{\Delta ADE}} = \frac {AB} {AD} \cdot \frac {AC} {AE}$$
		\begin{center}
			\input{figures/area_frac_plain.tex}
		\end{center}
	}
	%\block{Inscribed Angle Theorem}
	\block{Teorema dell'Angolo Inscritto}
	{
		$$\angle{AEB} = \angle{ADB} = \frac 1 2 \smallsmile{AB}$$
		\begin{center}
			\input{figures/duga_plain.tex}
		\end{center}
	}
	%\block{Angle Between a Tangent and a Chord}
	\block{Angolo tra una tangente e una corda}
	{
		$$\angle{ABC} = \angle{ADB} = \frac 1 2 \smallsmile{AB}$$
		\begin{center}
			\input{figures/kas_angle_plain.tex}
		\end{center}
	}
	%\block{Angle Formed by Two Chords}
	\block{Angolo formato da due corde}
	{
		$$\angle{AKB} = \angle{EKD} = \frac {\smallsmile{AB} \ + \smallsmile{ED}} {2}$$
		\begin{center}
			\input{figures/chord_angle_plain.tex}
		\end{center}
	}
	%\block{Angle Formed by Two Secants}
	\block{Angolo formaro da due Secanti}
	{
		$$\angle{BAC} = \frac {\smallsmile {DE} \ - \smallsmile{BC}} {2}$$
		\begin{center}
			\input{figures/sec_angle_plain.tex}
		\end{center}
	}
	
	\column{0.166}
	%\block{Angles Formed by Parallel Lines and a Transversal}
	\block{Angoli formati da linee parallele e da una secante}
	{
		$$k \ \parallelsum \ l \iff \alpha = \beta = \gamma$$
		\begin{center}
			\input{figures/parallel_plain.tex}
		\end{center}
	}
	%\block{Tangent Segments from a Common External Point}
	\block{Segmenti tangenti da un punto esterno comune}
	{
		$$AB = AC$$
		\begin{center}
			\input{figures/2kas_plain.tex}
		\end{center}
	}
	%\block{Tangent--Secant Theorem}
	\block{Teorema delle Tangenti delle Secanti}
	{
		$$\Delta ABC \sim \Delta ADB, \quad AB^2 = AC \cdot AD$$
		\begin{center}
			\input{figures/kas_plain.tex}
		\end{center}
	}
	%\block{Intersecting Chords Theorem}
	\block{Teorema delle Corde Secanti}
	{
		$$\Delta AKB \sim \Delta EKD, \quad AK \cdot KD = BK \cdot KE$$
		\vspace{0.5cm}
		\begin{center}
			\input{figures/chord_plain.tex}
		\end{center}
	}
	%\block{Secant--Secant Theorem}
	\block{Terorema delle Secanti}
	{
		$$\Delta ABC \sim \Delta ADE, \quad AB \cdot AD = AC \cdot AE$$
		\begin{center}
			\input{figures/sec_plain.tex}
		\end{center}
	}
	
	\column{0.166}
	%\block{Incenter of a Triangle}
	\block{Incentro di un Triangolo}
	{
		\begin{center}
			L'incentro, ottenuto dall'incrocio delle bisettrici, è sempre interno. È un punto equidistante da tutti i lati ed è il centro del cerchio inscritto \\
			\vspace{0.5cm}
			\input{figures/cvt_plain.tex}
		\end{center}
	}
	%\block{Circumcenter of a Triangle}
	\block{Circocentro di un Triangolo}
	{
		\begin{center}
			Il circocentro, ottenuto dall'incrocio degli assi,è equidistante dai vertici ed è il centro del cerchio circoscritto \\
			\vspace{0.5cm}
			\input{figures/cot_plain.tex}
		\end{center}
	}
	%\block{Circumcircle of a Right Triangle}
	\block{Circonferenza circoscritta a un Triangolo Rettangolo}
	{
		\begin{center}
			$\Delta ABC$ - Triangolo Rettangolo $\iff$ $AC$ - diametro,\\
			AO = OC = OB = R\\
			\vspace{0.5cm}
			\input{figures/ort_plain.tex}
		\end{center}
	}
	%\block{Cyclic Quadrilateral}
	\block{Quadrilatero Ciclico}
	{
		\begin{center}
			$ABCD$ - inscritto $\iff \alpha + \beta = 180^\circ$\\
			\vspace{0.5cm}
			\input{figures/v4_plain.tex}
		\end{center}
	}
	%\block{Tangential Quadrilateral}
	\block{Quadrilatico Circoscritto}
	{
		\begin{center}
			$ABCD$ - circoscritto $\iff AB + CD = BC + AD$\\
			\vspace{0.5cm}
			\input{figures/o4_plain.tex}
		\end{center}
	}
	
\end{columns}

\end{document}